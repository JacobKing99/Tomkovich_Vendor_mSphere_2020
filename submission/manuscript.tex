% Options for packages loaded elsewhere
\PassOptionsToPackage{unicode}{hyperref}
\PassOptionsToPackage{hyphens}{url}
%
\documentclass[
  11pt,
]{article}
\usepackage{lmodern}
\usepackage{amssymb,amsmath}
\usepackage{ifxetex,ifluatex}
\ifnum 0\ifxetex 1\fi\ifluatex 1\fi=0 % if pdftex
  \usepackage[T1]{fontenc}
  \usepackage[utf8]{inputenc}
  \usepackage{textcomp} % provide euro and other symbols
\else % if luatex or xetex
  \usepackage{unicode-math}
  \defaultfontfeatures{Scale=MatchLowercase}
  \defaultfontfeatures[\rmfamily]{Ligatures=TeX,Scale=1}
\fi
% Use upquote if available, for straight quotes in verbatim environments
\IfFileExists{upquote.sty}{\usepackage{upquote}}{}
\IfFileExists{microtype.sty}{% use microtype if available
  \usepackage[]{microtype}
  \UseMicrotypeSet[protrusion]{basicmath} % disable protrusion for tt fonts
}{}
\makeatletter
\@ifundefined{KOMAClassName}{% if non-KOMA class
  \IfFileExists{parskip.sty}{%
    \usepackage{parskip}
  }{% else
    \setlength{\parindent}{0pt}
    \setlength{\parskip}{6pt plus 2pt minus 1pt}}
}{% if KOMA class
  \KOMAoptions{parskip=half}}
\makeatother
\usepackage{xcolor}
\IfFileExists{xurl.sty}{\usepackage{xurl}}{} % add URL line breaks if available
\IfFileExists{bookmark.sty}{\usepackage{bookmark}}{\usepackage{hyperref}}
\hypersetup{
  hidelinks,
  pdfcreator={LaTeX via pandoc}}
\urlstyle{same} % disable monospaced font for URLs
\usepackage[margin=1.0in]{geometry}
\usepackage{graphicx,grffile}
\makeatletter
\def\maxwidth{\ifdim\Gin@nat@width>\linewidth\linewidth\else\Gin@nat@width\fi}
\def\maxheight{\ifdim\Gin@nat@height>\textheight\textheight\else\Gin@nat@height\fi}
\makeatother
% Scale images if necessary, so that they will not overflow the page
% margins by default, and it is still possible to overwrite the defaults
% using explicit options in \includegraphics[width, height, ...]{}
\setkeys{Gin}{width=\maxwidth,height=\maxheight,keepaspectratio}
% Set default figure placement to htbp
\makeatletter
\def\fps@figure{htbp}
\makeatother
\setlength{\emergencystretch}{3em} % prevent overfull lines
\providecommand{\tightlist}{%
  \setlength{\itemsep}{0pt}\setlength{\parskip}{0pt}}
\setcounter{secnumdepth}{-\maxdimen} % remove section numbering
\usepackage{helvet} % Helvetica font
\renewcommand*\familydefault{\sfdefault} % Use the sans serif version of the font
\usepackage[T1]{fontenc}

\usepackage[none]{hyphenat}

\usepackage{setspace}
\doublespacing
\setlength{\parskip}{1em}

\usepackage{lineno}

\usepackage{pdfpages}

\author{}
\date{\vspace{-2.5em}}

\begin{document}

\vspace{35mm}

\hypertarget{the-initial-gut-microbiota-and-response-to-antibiotic-perturbation-influence-clostridioides-difficile-clearance-in-mice}{%
\section{\texorpdfstring{The initial gut microbiota and response to
antibiotic perturbation influence \emph{Clostridioides difficile}
clearance in
mice}{The initial gut microbiota and response to antibiotic perturbation influence Clostridioides difficile clearance in mice}}\label{the-initial-gut-microbiota-and-response-to-antibiotic-perturbation-influence-clostridioides-difficile-clearance-in-mice}}

\vspace{35mm}

Sarah Tomkovich\({^1}\), Joshua M.A.~Stough\({^1}\), Lucas
Bishop\({^1}\), Patrick D. Schloss\textsuperscript{1\(\dagger\)}

\vspace{40mm}

\(\dagger\) To whom correspondence should be addressed:
\href{mailto:pschloss@umich.edu}{\nolinkurl{pschloss@umich.edu}}

\(1\) Department of Microbiology and Immunology, University of Michigan,
Ann Arbor, MI 48109

\newpage
\linenumbers

\hypertarget{abstract}{%
\subsection{Abstract}\label{abstract}}

The gut microbiota has a key role in determining susceptibility to
\emph{Clostridioides difficile} infections (CDIs). However, much of the
mechanistic work examining CDIs in mouse models use animals obtained
from a single source. We treated mice from 6 sources (2 University of
Michigan colonies and 4 commercial vendors) with clindamycin, followed
by a \emph{C. difficile} challenge and then measured \emph{C. difficile}
colonization levels throughout the infection. The microbiota were
profiled via 16S rRNA gene sequencing to examine the variation across
sources and alterations due to clindamycin treatment and \emph{C.
difficile} challenge. While all mice were colonized 1-day
post-infection, variation emerged from days 3-7 post-infection with
animals from some sources colonized with \emph{C. difficile} for longer
and at higher levels. We identified bacteria that varied in relative
abundance across sources and throughout the experiment. Some bacteria
were consistently impacted by clindamycin treatment in all sources of
mice including \emph{Lachnospiraceae}, \emph{Ruminococcaceae}, and
\emph{Enterobacteriaceae}. To identify bacteria that were most important
to colonization regardless of the source, we created logistic regression
models that successfully classified mice based on whether they cleared
\emph{C. difficile} by 7 days post-infection using community composition
data at baseline, post-clindamycin, and 1-day post-infection. With these
models, we identified 4 bacteria that were predictive of whether
\emph{C. difficile} cleared. They varied across sources
(\emph{Bacteroides}), were altered by clindamycin
(\emph{Porphyromonadaceae}), or both (\emph{Enterobacteriaceae} and
\emph{Enterococcus}). Allowing for microbiota variation across sources
better emulates human inter-individual variation and can help identify
bacterial drivers of phenotypic variation in the context of CDIs.

\hypertarget{importance}{%
\subsection{Importance}\label{importance}}

\emph{Clostridioides difficile} is a leading nosocomial infection.
Although perturbation to the gut microbiota is an established risk,
there is variation in who becomes asymptomatically colonized, develops
an infection, or has adverse infection outcomes. Mouse models of
\emph{C. difficile} infection (CDI) are widely used to answer a variety
of \emph{C. difficile} pathogenesis questions. However, the
inter-individual variation between mice from the same breeding facility
is less than what is observed in humans. Therefore, we challenged mice
from 6 different breeding colonies with \emph{C. difficile}. We found
that the starting microbial community structures and \emph{C. difficile}
persistence varied by the source of mice. Interestingly, a subset of the
bacteria that varied across sources were associated with how long
\emph{C. difficile} was able to colonize. By increasing the
inter-individual diversity of the starting communities, we were able to
better model human diversity. This provided a more nuanced perspective
of \emph{C. difficile} pathogenesis.

\newpage

\hypertarget{introduction}{%
\subsection{Introduction}\label{introduction}}

Antibiotics are a common risk factor for \emph{Clostridioides difficile}
infections (CDIs) due to their effect on the intestinal microbiota, but
there is variation in who goes on to develop severe or recurrent CDIs
after exposure (1, 2). Additionally, asymptomatic colonization, where
\emph{C. difficile} is detectable, but symptoms are absent, has been
documented in infants and adults (3, 4). The intestinal microbiota has
been implicated in asymptomatic colonization (5, 6), susceptibility to
CDIs (7), and adverse CDI outcomes (9--12). However, it is not clear how
much inter-individual microbiota variation contributes to the range of
outcomes observed after \emph{C. difficile} exposure relative to other
risk factors.

Mouse models of CDIs have been a great tool for understanding \emph{C.
difficile} pathogenesis (13). The number of CDI mouse model studies has
grown substantially since Chen et al.~published their C57BL/6 model in
2008, which disrupted the gut microbiota with antibiotics to enable
\emph{C. difficile} colonization and symptoms such as diarrhea and
weight loss (14). CDI mouse models have been used to examine
translationally relevant questions regarding \emph{C. difficile},
including the role of the microbiota and efficacy of potential
therapeutics for treating CDIs (15). However, variation in the
microbiota between mice from the same breeding colony is much less than
the inter-individual variation observed between humans (16, 17).
Studying CDIs in mice with a homogeneous microbiota is likely to
overstate the importance of individual mechanisms. Using mice that have
a more heterogeneous microbiota would allow researchers to identify and
validate more generalizable mechanisms responsible for CDI.

In the past, our group has attempted to introduce more variation into
the mouse microbiota by using a variety of antibiotic treatments
(18--21). An alternative approach to maximize microbiota variation is to
use mice from multiple sources (22, 23). The differences between the
microbiota of mice from vendors have been well documented and shown to
influence susceptibility to a variety of diseases (24, 25), including
enteric infections (22, 23, 26--30). Different research groups have also
observed different CDI outcomes despite using similar murine models (13,
18, 21, 31--33). Here we examined how variation in the baseline
microbiota and responses to clindamycin treatment in C57BL/6 mice from
six different sources influenced susceptibility to \emph{C. difficile}
colonization and the time needed to clear the infection.

\hypertarget{results}{%
\subsection{Results}\label{results}}

\textbf{The variation in the microbiota is high between mice from
different sources.} We obtained C57BL/6 mice from 6 different sources:
two colonies from the University of Michigan that were split from each
other in 2010 (the Young and Schloss lab colonies) and four commercial
vendors: the Jackson Laboratory, Charles River Laboratories, Taconic
Biosciences, and Envigo (which was formerly Harlan). These 4 vendors
were chosen because they are commonly used for murine CDI studies (26,
34--40). Two experiments were conducted, approximately 3 months apart.

We sequenced the V4 region of the 16S rRNA gene from fecal samples
collected from these mice after they acclimated to the University of
Michigan animal housing environment. We first examined the alpha
diversity across the 6 sources of mice. There was a significant
difference in the richness (i.e.~number of observed operational
taxonomic units (OTUs)), but not Shannon diversity index across the
sources of mice (\emph{P}\textsubscript{FDR} = 0.03 and
\emph{P}\textsubscript{FDR} = 0.052, respectively; Fig. 1A-B and Data
Set S1, Sheets 1-2). Next, we compared the community structure of mice
(Fig. 1C). The source of mice and the interactions between the source
and cage effects explained most of the observed variation between fecal
communities (PERMANOVA combined R\textsuperscript{2} = 0.90, \emph{P}
\textless{} 0.001; Fig. 1C and Data Set S1, Sheet 3). Mice that are
co-housed tend to have similar gut microbiotas due to coprophagy (41).
Since mice within the same source were housed together, it was not
surprising that the cage effect also contributed to the observed
community variation. There were some differences between the 2
experiments we conducted, as the experiment and cage effects
significantly explained the observed community variation for the Schloss
and Young lab mouse colonies (Fig. S1A-B and Data Set S1, Sheet 4).
However, most of the vendors also clustered by experiment (Fig. S1C-D,
F), suggesting there was some community variation between the 2
experiments within each source, particularly for Schloss, Young, and
Envigo mice (Fig. S1G-H). After finding differences at the community
level, we next identified the bacteria that varied between sources of
mice. There were 268 OTUs with relative abundances that were
significantly different between the sources at baseline (Fig. 1D and
Data Set S1, Sheet 5). Though we saw differences between experiments at
the community level, there were no OTUs that were significantly
different between experiments within Schloss, Young, and Envigo mice at
baseline (all \emph{P} \textgreater{} 0.05). By using mice from six
sources we were able to increase the variation in the starting
communities to evaluate in a clindamycin-based CDI model.

\textbf{Clindamycin treatment renders all mice susceptible to \emph{C.
difficile} 630 colonization, but clearance time varies across sources.}
Clindamycin is frequently implicated with human CDIs (42) and was part
of the antibiotic treatment for the frequently cited 2008 CDI mouse
model (14). We have previously demonstrated mice are rendered
susceptible to \emph{C. difficile}, but clear the pathogen within 9
days, thus colonization is transient when treated with clindamycin alone
(21, 43). All mice were treated with 10 mg/kg clindamycin via
intraperitoneal injection and one day later challenged with
10\textsuperscript{3} \emph{C. difficile} 630 spores (Fig. 2A). The day
after infection, \emph{C. difficile} was detectable in all mice at a
similar level (median CFU range: 2.2e+07-1.3e+08;
\emph{P}\textsubscript{FDR} = 0.15), indicating clindamycin rendered all
mice susceptible regardless of source (Fig. 2B). However, between 3 and
7 days post-infection, we observed variation in \emph{C. difficile}
levels across sources of mice (all \emph{P}\textsubscript{FDR} \(\le\)
0.019; Fig. 2B and Data Set S1, Sheet 6). This suggested the source of
mice was associated with \emph{C. difficile} clearance. While the
colonization dynamics were similar between the two experiments, the
Schloss mice took longer to clear \emph{C. difficile} in the first
experiment compared to the second and the Envigo mice took longer to
clear \emph{C. difficile} in the second experiment compared to the first
(Fig. S2A-B). The change in the mice's weight significantly varied
across sources of mice with the most weight loss occurring two days
post-infection (Fig. 2C and Data Set S1, Sheet 7). There was also one
Jackson and one Envigo mouse that died between 1- and 3-days
post-infection during the second experiment. Mice obtained from Jackson,
Taconic, and Envigo tended to lose more weight, have higher \emph{C.
difficile} CFU levels and take longer to clear the infection compared to
the other sources of mice (although there was variation between
experiments with Schloss and Envigo mice). This was particularly evident
7 days post-infection (Fig. 2B-C, Fig. S2C-D), when 57\% of the mice
were still colonized with \emph{C. difficile} (Fig. S2E). By 9 days
post-infection the majority of the mice from all sources had cleared
\emph{C. difficile} with the exception of 1 Taconic mouse from the first
experiment and 2 Envigo mice from the second experiment (Fig. 2B). Thus,
clindamycin rendered all mice susceptible to \emph{C. difficile} 630
colonization, regardless of source, but there was significant variation
in disease phenotype across the sources of mice.

\textbf{Clindamycin treatment alters bacteria in all sources, but a
subset of bacterial differences across sources persists.} Given the
variation in fecal communities that we observed across breeding
colonies, we hypothesized that variation in \emph{C. difficile}
clearance would be explained by community variation across the 6 sources
of mice. As expected, clindamycin treatment decreased the richness and
Shannon diversity across all sources of mice (Fig. 3A-B). Interestingly,
significant differences in diversity metrics between sources emerged
after clindamycin treatment, with Charles River mice having higher
richness and Shannon diversity than most of the other sources
(\emph{P}\textsubscript{FDR} \textless{} 0.05; Fig 3A-B and Data Set S1,
Sheets 1-2). The clindamycin treatment decreased the variation in
community structures between sources of mice. The source of mice and the
interactions between source and cage effects explained almost all of the
observed variation between communities (combined R\textsuperscript{2} =
0.99, \emph{P} \textless{} 0.001; Fig. 3C and Data Set S1, Sheet 3).
However, there were only 18 OTUs with relative abundances that
significantly varied between sources after clindamycin treatment (Fig.
3D and Data Set S1, Sheet 8). Next, we identified the bacteria that
shifted after clindamycin treatment, regardless of source by analyzing
paired fecal samples from mice that were collected at baseline and after
clindamycin treatment. We identified 153 OTUs that were altered after
clindamycin treatment in most mice (Fig. 3E and Data Set S1, Sheet 9).
When we compared the list of significant clindamycin impacted bacteria
with the bacteria that varied between sources post-clindamycin, we found
4 OTUs that were shared between the lists (\emph{Enterobacteriaceae}
(OTU 1), \emph{Lachnospiraceae} (OTU 130), \emph{Lactobacillus} (OTU 6),
\emph{Enterococcus} (OTU 23); Fig. 3D-E and Data Set S1, Sheets 8-9).
Importantly, some of the OTUs that varied between sources also shifted
with clindamycin treatment. For example, \emph{Proteus} increased after
clindamycin treatment (Fig. 3D), but only in Taconic mice.
\emph{Enterococcus} was primarily found in mice purchased from
commercial vendors and also increased in relative abundance after
clindamycin treatment (Fig. 3D). These findings demonstrate that
clindamycin had a consistent impact on the fecal bacterial communities
of mice from all sources and only a subset of the OTUs continued to vary
between sources.

\textbf{Microbiota variation between sources is maintained after
\emph{C. difficile} challenge.} One day post-infection, significant
differences in diversity metrics remained across sources
(\emph{P}\textsubscript{FDR} \textless{} 0.05, Fig 4A-B and Data Set S1,
Sheets 1-2). Although the Charles River mice had more diverse
communities and were also able to clear \emph{C. difficile} faster than
the other sources, diversity did not explain the observed variation in
\emph{C. difficile} colonization across sources. The Young and Schloss
mice had the lowest diversity 1 day post-infection and were able to
clear \emph{C. difficile} earlier than Jackson, Taconic and Envigo mice.
The source of mice and the interactions between source and cage effects
continued to explain most of the observed community variation (combined
R\textsuperscript{2} = 0.88; \emph{P} \textless{} 0.001; Fig. 4C and
Data Set S1, Sheet 3). One day after \emph{C. difficile} challenge,
there were 44 OTUs with significantly different relative abundances
across sources (Fig. 4D and Data Set S1, Sheet 10).

Throughout the experiment, the source of mice continued to be the
dominant factor that explained the observed variation across fecal
communities (PERMANOVA R\textsuperscript{2} = 0.35, \emph{P} \textless{}
0.001) followed by interactions between cage effects and the day of the
experiment (Movie S1 and Data Set S1, Sheet 11). Fecal samples from the
same source of mice continued to cluster closely to each other
throughout the experiment. By 7 days post-infection, when approximately
43\% mice had cleared \emph{C. difficile}, most of the mice had not
recovered to their baseline community structure (Fig. 4E). The distance
to the baseline community did not explain the variation in \emph{C.
difficile} clearance as the Schloss and Young mice had mostly cleared
\emph{C. difficile}, but their communities were a greater distance from
baseline 7 days post-infection compared to the Jackson and Taconic mice
that were still colonized. In summary, mouse bacterial communities
varied significantly between sources throughout the course of the
experiment and a consistent subset of bacteria remained different
between sources regardless of clindamycin and \emph{C. difficile}
challenge.

\textbf{Baseline, post-clindamycin, and post-infection community data
can predict mice that will clear \emph{C. difficile} by 7 days
post-infection.} After identifying taxa that varied between sources,
changed after clindamycin treatment, or both, we determined which taxa
were influencing the variation in \emph{C. difficile} colonization at
day 7 (Fig. 2B, Fig. S2C). We trained three L2-regularized logistic
regression models with either input bacterial community data from the 6
sources of mice at the baseline (day = -1), post-clindamycin (day = 0),
or post-infection (day = 1) timepoints of the experiment to predict
\emph{C. difficile} colonization status on day 7 (Fig. S3A-B). All
models were better at predicting \emph{C. difficile} colonization status
on day 7 than random chance (all \emph{P} \textless{} 0.001, Data Set
S1, Sheet 12). The model based on the post-clindamycin (AUROC = 0.78)
community OTU data performed significantly better than the baseline
(AUROC = 0.72) or the post-infection (AUROC = 0.67) models
(\emph{P}\textsubscript{FDR} \textless{} 0.001 for pairwise comparisons;
Fig. S3C and Data Set S1, Sheet 13). Thus, we were able to use bacterial
relative abundance data from the time of \emph{C. difficile} challenge
to differentiate mice that had cleared \emph{C. difficile} before day 7
from the mice still colonized with \emph{C. difficile} at that
timepoint. This result suggests that the bacterial community's response
to clindamycin treatment had the greatest influence on subsequent
\emph{C. difficile} colonization dynamics.

To examine the bacteria that were driving each model's performance, we
selected the 20 OTUs that had the highest absolute feature weights in
each of the 3 models (Data Set S1, Sheet 14). First, we looked at OTUs
from the model with the best performance, which was based on the
post-clindamycin treatment (day 0) bacterial community data. Out of the
10 highest ranked OTUs, 7 OTUs were associated with \emph{C. difficile}
colonization 7 days post-infection (\emph{Bacteroides},
\emph{Escherichia/Shigella}, 2 \emph{Lachnospiraceae},
\emph{Lactobacillus}, \emph{Porphyromonadaceae}, and
\emph{Ruminococcaceae}), while 3 OTUs were associated with clearance
(\emph{Enterobacteriaceae}, \emph{Lachnospiraceae},
\emph{Porphyromonadaceae}; Fig. 5A). On day 0, the majority of these
OTUs were impacted by clindamycin and had relative abundances that were
close to the limit of detection (Fig. 5A). Next, we examined whether any
of the top 20 ranked OTUs from the post-clindamycin (day 0) model were
also important in the other 2 classification models based on baseline
(day -1) and 1 day post-infection community data. We identified 6 OTUs
that were important to the post-clindamycin model and either the
baseline or 1 day post-infection models (\emph{Enterobacteriaceae,
Ruminococcaceae, Lactobacillus, Bacteroides, Porphyromonadaceae,
Erysipelotrichaceae}; Data Set S1, Sheet 14). Thus, a subset of
bacterial OTUs were important for determining \emph{C. difficile}
colonization dynamics across multiple timepoints.

To determine whether the OTUs driving the classification models also
varied between sources, were altered by clindamycin treatment, or both,
we identified the OTUs from each model that varied between sources (Fig.
1D, 3D, 4D and Data Set S1, Sheets 5, 8, and 10) or were impacted by
clindamycin treatment (Fig. 3E and Data Set S1, Sheet 9; Fig. S4).
Comparing the features important to the 3 models identified 14 OTUs
associated with source, 21 OTUs associated with clindamycin treatment,
and 6 OTUs associated with both (Fig. 5B). Together, these results
suggest that the initial bacterial communities and their responses to
clindamycin influenced the clearance of \emph{C. difficile}.

Several OTUs that overlapped with our previous analyses appeared across
at least 2 models (\emph{Bacteroides, Enterococcus, Enterobacteriaceae,
Porphyromonadaceae}), so we examined how the relative abundances of
these OTUs varied over the course of the experiment (Fig. 6). Across the
9 days post-infection, there was at least 1 timepoint when the relative
abundances of these OTUs significantly varied between sources (Data Set
S1, Sheet 15). Interestingly, there were no OTUs that emerged as
consistently enriched or depleted in mice that were colonized past 7
days post-infection, suggesting that multiple bacteria influence
\emph{C. difficile} colonization dynamics.

\hypertarget{discussion}{%
\subsection{Discussion}\label{discussion}}

Applying our CDI model to 6 different sources of mice, allowed us to
identify bacterial taxa that were unique to different sources as well as
taxa that were universally impacted by clindamycin. We trained
L2-regularized logistic regression models with baseline (day -1),
post-clindamycin treatment (day 0), and 1-day post-infection fecal
community data that could predict whether mice cleared \emph{C.
difficile} by 7 days post-infection better than random chance. We
identified \emph{Bacteroides, Enterococcus, Enterobacteriaceae,
Porphyromonadaceae} (Fig. 6) as candidate bacteria within these
communities that influenced variation in \emph{C. difficile}
colonization dynamics since these bacteria were all important in the
logistic regression models and varied by source, were impacted by
clindamycin treatment, or both. Overall, our results demonstrated
clindamycin was sufficient to render mice from multiple sources
susceptible to CDI and only a subset of the inter-individual microbiota
variation across mice from different sources was needed to predict which
mice could clear \emph{C. difficile}.

Other studies have used mice from multiple sources to identify bacteria
that either promote colonization resistance or increase susceptibility
to enteric infections (22, 23, 26--30). For example, against
\emph{Salmonella} infections, \emph{Enterobacteriaceae} and segmented
filamentous bacteria have emerged as protective (22, 27). We found
\emph{Enterobacteriaceae} increased in all sources of mice after
clindamycin treatment, positively correlating with \emph{C. difficile}
colonization. However, there was also variation in
\emph{Enterobacteriaceae} relative abundance levels between sources that
was associated with the variation in \emph{C. difficile} colonization
dynamics across sources. Thus, bacteria may have differential roles in
determining susceptibility depending on the type of bacterial infection.

Differences in CDI mouse model studies have been attributed to
intestinal microbiota variation across sources. For example, researchers
using the same clindamycin treatment and C57BL/6 mice had different
\emph{C. difficile} outcomes, one having sustained colonization (32),
while the other had transient colonization (18), despite both using
\emph{C. difficile} VPI 10643. Baseline differences in the microbiota
composition have been hypothesized to partially explain the differences
in colonization outcomes and overall susceptibility to \emph{C.
difficile} after treatment with the same antibiotic (13, 31). When we
treated mice from 6 different sources with clindamycin and challenged
them with \emph{C. difficile} 630, we found microbiota variation across
sources impacted colonization outcomes, but not susceptibility. A
previous study with \emph{C. difficile} identified an endogenous
protective \emph{C. difficile} strain LEM1 that bloomed after antibiotic
treatment in mice from Jackson or Charles River Laboratories, but not
Taconic that protected mice against the more toxigenic \emph{C.
difficile} VPI10463 (26). Given that we obtained mice from the same
vendors, we checked all mice for endogenous \emph{C. difficile} by
plating stool samples that were collected after clindamycin treatment.
However, we did not identify any endogenous \emph{C. difficile} strains
prior to challenge, suggesting there were no endogenous protective
strains in the mice we received and other bacteria mediated the
variation in \emph{C. difficile} colonization across sources. The
\emph{C. difficile} strain used could also be contributing to the
variation in \emph{C. difficile} outcomes seen across different research
groups. For example, a group found differential colonization outcomes
after clindamycin treatment, with \emph{C. difficile} 630 and M68
infections eventually becoming undetectable while strain BI-7 remained
detectable up to 70 days post-treatment (44). One study limitation is
that we only used female mice. Sex has been shown to influence
microbiota variation in mice (45), so we used female mice to reduce this
confounding variable and also match the sex used in previous CDI studies
that administered clindamycin to mice (32, 33, 44, 46). The bacterial
perturbations induced by clindamycin treatment have been well
characterized and our findings agree with previous CDI mouse model work
demonstrating \emph{Enterococcus} and \emph{Enterobacteriaceae} were
associated with \emph{C. difficile} susceptibility and
\emph{Porpyhromonadaceae}, \emph{Lachnospiraceae},
\emph{Ruminococcaceae}, and \emph{Turicibacter} were associated with
resistance (19, 21, 32, 33, 43, 44, 46, 47). While we have demonstrated
that susceptibility is uniform across sources of mice after clindamycin
treatment, there could be different outcomes for either susceptibility
or clearance in the case of other antibiotic treatments.

We found the time needed to naturally clear \emph{C. difficile} varied
across sources of mice implying that at least in the context of the same
perturbation, microbiota differences influence infection outcome. More
importantly, we were able to explain the variation observed across
sources with a subset of OTUs that were also important for predicting
\emph{C. difficile} colonization status 7 days post-infection. Since all
but 3 mice eventually cleared \emph{C. difficile} 630 by 9 days
post-infection and the model built with the post-clindamycin (day 0) OTU
relative abundance data had the best performance, our results suggest
clindamycin treatment had a larger role in determining \emph{C.
difficile} susceptibility and clearance than the source of the mice.

Using mice from multiple sources successfully increased the inter-animal
variation. One alternative approach that has been used in some CDI
studies is to associate mice with human microbiotas (48--53). However, a
major caveat to this method is the substantial loss of human microbiota
community members upon transfer to mice (54, 55). Additionally, with the
exception of 2 recent studies (48, 49), most of these studies associated
mice with just 1 type of human microbiota either from a single donor or
a single pool from multiple donors (50--53). This approach does not aid
in the goal of modeling the interpersonal variation seen in humans to
understand how the microbiota influences susceptibility to CDIs and
adverse outcomes. Importantly, our study using mice from 6 different
sources increased the variation between groups of mice compared to using
1 source alone, to better reflect the inter-individual microbiota
variation observed in humans.

Another motivation for associating mice with human microbiotas is to
study the bacteria associated with the disease in humans. Decreased
\emph{Bifidobacterium}, \emph{Porphyromonas}, \emph{Ruminococcaceae} and
\emph{Lachnospiraceae} and increased \emph{Enterobacteriaceae},
\emph{Enterococcus}, \emph{Lactobacillus}, and \emph{Proteus} have all
been associated with human CDIs (7). Encouragingly, these populations
were well represented in our study, suggesting most of the mouse sources
are suitable for gaining insights into the bacteria influencing \emph{C.
difficile} colonization and infections in humans. An important exception
was \emph{Enterococcus}, which was primarily absent from University of
Michigan colonies and \emph{Proteus}, which was only found in Taconic
mice. The fact that some CDI-associated bacteria were only found in a
subset of mice has important implications for future CDI mouse model
studies, but also models the natural patchiness of microbial populations
in humans.

Other microbiota and host factors that were outside the scope of our
current study may also contribute to the differences in \emph{C.
difficile} colonization dynamics between sources of mice. The microbiota
is composed of viruses, fungi, and parasites in addition to bacteria,
and these non-bacterial members can also vary across sources of mice
(56, 57). While our study focused solely on the bacterial portion,
viruses and fungi have also begun to be implicated in the context of
CDIs or FMT treatments for recurrent CDIs (35, 58--61). Beyond community
composition, the metabolic function of the microbiota also has a CDI
signature (20, 47, 62, 63) and can vary across mice from different
sources (64). For example, microbial metabolites, particularly secondary
bile acids and butyrate production, have been implicated as important
contributors to \emph{C. difficile} resistance (33, 44). Interestingly,
butyrate has previously been shown to vary across mouse vendors and
mediated resistance to \emph{Citrobacter rodentium} infection, a model
of enterohemorrhagic and enteropathogenic \emph{Escherichia coli}
infections (23). Evidence for immunological toning differences in IgA
and Th17 cells across mice from different vendors have also been
documented (65, 66) and could influence the host response to CDI (67,
68), particularly relevant for \emph{C. difficile} strains that induce
more severe disease than \emph{C. difficile} 630. The outcome after
\emph{C. difficile} exposure depends on a multitude of factors,
including genetics, age, diet, and immunity; all of which also influence
the microbiota.

We have demonstrated that the ways baseline microbiotas from different
mouse sources respond to clindamycin treatment influence the length of
time mice remained colonized with \emph{C. difficile} 630. To better
understand the contribution of the microbiota to \emph{C. difficile}
pathogenesis and treatments, using multiple sources of mice may yield
more insights than a single source. Furthermore, for studies wanting to
examine the interplay between particular bacteria such as
\emph{Enterococcus} and \emph{C. difficile}, these results could serve
as a resource for selecting mice to address the question. Using mice
from multiple sources helps model the interpersonal microbiota variation
among humans to aid our understanding of how the gut microbiota provides
colonization resistance to CDIs.

\newpage

\hypertarget{acknowledgements}{%
\subsection{Acknowledgements}\label{acknowledgements}}

This work was supported by the National Institutes of Health
(U01AI124255). ST was supported by the Michigan Institute for Clincial
and Health Research Postdoctoral Translation Scholars Program
(UL1TR002240 from the National Center for Advancing Translational
Sciences). We thank members of the Schloss lab for feedback on planning
the experiments and data presentation. In particular, we thank Begüm
Topçuoğlu for help with implementing logistic regression models, Ana
Taylor for help with media preparation and sample collection, and
Nicholas Lesniak for his critical feedback on the manuscript. We also
thank members of Vincent Young's lab, particularly Kimberly Vendrov, for
guidance with the \emph{C. difficile} infection mouse model and donating
the mice. We also thank the Unit for Laboratory Animal Medicine at the
University of Michigan for maintaining our mouse colony and providing
the institutional support for our mouse experiments. Finally, we thank
Kwi Kim, Austin Campbell, and Kimberly Vendrov for their help in
maintaining the Schloss lab's anaerobic chamber.

\newpage

\hypertarget{materials-and-methods}{%
\subsection{Materials and Methods}\label{materials-and-methods}}

\textbf{(i) Animals.} All experiments were approved by the University of
Michigan Animal Care and Use Committee (IACUC) under protocol number
PRO00006983. Female C57BL/7 mice were obtained from 6 different sources:
The Jackson Laboratory, Charles River Laboratories, Taconic Biosciences,
Envigo, and two colonies at the University of Michigan (the Schloss lab
colony and the Young lab colony). The Young lab colony was originally
established with mice purchased from Jackson in 2002, and the Schloss
lab colony was established in 2010 with mice donated from the Young lab.
The 4 groups of mice purchased from vendors were allowed to acclimate to
the University of Michigan mouse facility for 13 days prior to starting
the experiment. At least 4 female mice (age 5-10 weeks) were obtained
per source and mice from the same source were primarily housed at a
density of 2 mice per cage. The experiment was repeated once,
approximately 3 months after the start of the first experiment.

\textbf{(ii) Antibiotic treatment.} After the 13-day acclimation period,
all mice received 10 mg/kg clindamycin (filter sterilized through a 0.22
micron syringe filter prior to administration) via intraperitoneal
injection (Fig. 1A).

\textbf{(iii) \emph{C. difficile} infection model.} Mice were challenged
with 10\textsuperscript{3} spores of \emph{C. difficile} strain 630 via
oral gavage post-infection 1 day after clindamycin treatment as
described previously (21). Mice weights and stool samples were taken
daily through 9 days post-infection (Fig. 1A). Collected stool was split
for \emph{C. difficile} quantification and 16S rRNA sequencing analysis.
For \emph{C. difficile} quantification, stool samples were transferred
to the anaerobic chamber, serially diluted in PBS, plated on
taurocholate-cycloserine-cefoxitin-fructose agar (TCCFA) plates, and
counted after 24 hours of incubation at 37°C under anaerobic conditions.
A sample from the day 0 timepoint (post-clindamycin and prior to
\emph{C. difficile} challenge) was also plated on TCCFA to ensure mice
were not already colonized with \emph{C. difficile} prior to infection.
There were 3 deaths recorded over the course of the experiment, 1
Taconic mouse died prior to \emph{C. difficile} challenge and 1 Jackson
and 1 Envigo mouse died between 1- and 3-days post-infection. Mice were
categorized as cleared when no \emph{C. difficile} was detected in the
first serial dilution (limit of detection: 100 CFU). Stool samples for
16S rRNA sequencing were snap frozen in liquid nitrogen and stored at
-80°C until DNA extraction.

\textbf{(iv) 16S rRNA sequencing.} DNA was extracted from -80°C stored
stool samples using the DNeasy Powersoil HTP 96 kit (Qiagen) and an
EpMotion 5075 automated pipetting system (Eppendorf). The V4 region was
amplified for 16S rRNA with the AccuPrime Pfx DNA polymerase (Thermo
Fisher Scientific) using custom barcoded primers, as previously
described (69). The ZymoBIOMICS microbial community DNA standards was
used as a mock community control (70) and water was used as a negative
control per 96-well extraction plate. The PCR amplicons were cleaned up
and normalized with the SequalPrep normalization plate kit (Thermo
Fisher Scientific). Amplicons were pooled and quantified with the KAPA
library quantification kit (KAPA biosystems), prior to sequencing using
the MiSeq system (Illumina).

\textbf{(v) 16S rRNA gene sequence analysis.} mothur (v. 1.43) was used
to process all sequences (71) with a previously published protocol (69).
Reads were combined and aligned with the SILVA reference database (72).
Chimeras were removed with the VSEARCH algorithm and taxonomic
assignment was completed with a modified version (v16) of the Ribosomal
Database Project reference database (v11.5) (73) with an 80\% confidence
cutoff. Operational taxonomic units (OTUs) were assigned with a 97\%
similarity threshold using the opticlust algorithm (74). Based on the
mock communities, our overall sequencing error rate was 0.0112\% and all
water controls had less than 1000 sequences (range: 18-875). To account
for uneven sequencing across samples, samples were rarefied to 5,437
sequences 1,000 times for alpha and beta diversity analyses, and a
single time to generate relative abundances for model training. PCoAs
were generated based on the Yue and Clayton measure of dissimilarity
(\(\theta_{YC}\)) distances (75). Permutational multivariate analysis of
variance (PERMANOVA) was performed on mothur-generated \(\theta_{YC}\)
distance matrices with the adonis function in the vegan package (76) in
R (77).

\textbf{(vi) Classification model training and evaluation.} Models were
generated based on mice that were categorized as either cleared or
colonized 7 days post-infection and had sequencing data from the
baseline (day -1), post-clindamycin (day 0), and post-infection (day 1)
timepoints of the experiment. Input bacterial community relative
abundance data at the OTU level from the baseline, post-clindamycin, and
1-day post-infection timepoints was used to generate 3 classification
models that predicted \emph{C. difficile} colonization status 7 days
post-infection. The L2-regularized logistic regression models were
trained and tested using the caret package (78) in R as previously
described (79) with the exception that we used 60\% training and 40\%
testing data splits for testing of the held out test data to measure
model performance and repeated 2.5-fold cross-validation of the training
data to select the best cost hyperparameter. The modified training to
testing ratio was selected to accommodate the small number of samples in
the dataset. Code was modified from
\url{https://github.com/SchlossLab/ML_pipeline_microbiome} to update the
classification outcomes and change the data split ratios. The modified
repository to regenerate our modeling analysis is available at
\url{https://github.com/tomkoset/ML_pipeline_microbiome}.

\textbf{(vii) Statistical analysis.} All statistical tests were
performed in R (v 4.0.2) (77). The Kruskal-Wallis test was used to
analyze differences in \emph{C. difficile} CFU, mouse weight change, and
alpha diversity across sources with a Benjamini-Hochberg correction for
testing multiple timepoints, followed by pairwise Wilcoxon comparisons
with Benjamini-Hochberg correction. For taxonomic analysis and
generation of logistic regression model input data, \emph{C. difficile}
(OTU 20) was removed. Bacterial relative abundances that varied across
sources at the OTU level were identified with the Kruskal-Wallis test
with Benjamini-Hochberg correction for testing all identified OTUs,
followed by pairwise Wilcoxon comparisons with Benjamini-Hochberg
correction. The Wilcoxon rank sum test was used to test for OTUs that
differed between experiments within the Schloss, Young, and Envigo
sources with Benjamini-Hochberg correction for testing all identified
OTUs. OTUs impacted by clindamycin treatment were identified using the
paired Wilcoxon signed rank test with matched pairs of mice samples from
day -1 and day 0. To determine whether classification models had better
performance (test AUROCs) than random chance (0.5), we used the
one-sample Wilcoxon signed rank test. To examine whether there was an
overall difference in predictive performance across the 3 classification
models we used the Kruskal-Wallis test followed by pairwise Wilcoxan
comparisons with Benjamini-Hochberg correction for multiple hypothesis
testing. The tidyverse package (v 1.3.0) was used to wrangle and graph
data (80).

\textbf{(viii) Code availability.} Code for all data analysis and
generating this manuscript is available at
\url{https://github.com/SchlossLab/Tomkovich_Vendor_mSphere_2020}.

\textbf{(ix) Data availability.} The 16S rRNA sequencing data have been
deposited in the National Center for Biotechnology Information Sequence
Read Archive (BioProject Accession no. PRJNA608529).

\newpage

\hypertarget{references}{%
\subsection{References}\label{references}}

\hypertarget{refs}{}
\leavevmode\hypertarget{ref-Teng2019}{}%
1. Teng C, Reveles KR, Obodozie-Ofoegbu OO, Frei CR. 2019.
\emph{Clostridium difficile} infection risk with important antibiotic
classes: An analysis of the FDA adverse event reporting system.
International Journal of Medical Sciences 16:630--635.

\leavevmode\hypertarget{ref-Kelly2012}{}%
2. Kelly CP. 2012. Can we identify patients at high risk of recurrent
\emph{Clostridium difficile} infection? Clinical Microbiology and
Infection 18:21--27.

\leavevmode\hypertarget{ref-Zacharioudakis2015}{}%
3. Zacharioudakis IM, Zervou FN, Pliakos EE, Ziakas PD, Mylonakis E.
2015. Colonization with toxinogenic \emph{C. Difficile} upon hospital
admission, and risk of infection: A systematic review and meta-analysis.
American Journal of Gastroenterology 110:381--390.

\leavevmode\hypertarget{ref-Crobach2018}{}%
4. Crobach MJT, Vernon JJ, Loo VG, Kong LY, Péchiné S, Wilcox MH,
Kuijper EJ. 2018. Understanding \emph{Clostridium difficile}
colonization. Clinical Microbiology Reviews 31.

\leavevmode\hypertarget{ref-Zhang2015}{}%
5. Zhang L, Dong D, Jiang C, Li Z, Wang X, Peng Y. 2015. Insight into
alteration of gut microbiota in \emph{Clostridium difficile} infection
and asymptomatic c. Difficile colonization. Anaerobe 34:1--7.

\leavevmode\hypertarget{ref-VanInsberghe2020}{}%
6. VanInsberghe D, Elsherbini JA, Varian B, Poutahidis T, Erdman S, Polz
MF. 2020. Diarrhoeal events can trigger long-term \emph{Clostridium
difficile} colonization with recurrent blooms. Nature Microbiology
5:642--650.

\leavevmode\hypertarget{ref-Mancabelli2017}{}%
7. Mancabelli L, Milani C, Lugli GA, Turroni F, Cocconi D, Sinderen D
van, Ventura M. 2017. Identification of universal gut microbial
biomarkers of common human intestinal diseases by meta-analysis. FEMS
Microbiology Ecology 93.

\leavevmode\hypertarget{ref-Duvallet2017}{}%
8. Duvallet C, Gibbons SM, Gurry T, Irizarry RA, Alm EJ. 2017.
Meta-analysis of gut microbiome studies identifies disease-specific and
shared responses. Nature Communications 8.

\leavevmode\hypertarget{ref-Seekatz2016}{}%
9. Seekatz AM, Rao K, Santhosh K, Young VB. 2016. Dynamics of the fecal
microbiome in patients with recurrent and nonrecurrent \emph{Clostridium
difficile} infection. Genome Medicine 8.

\leavevmode\hypertarget{ref-Khanna2016}{}%
10. Khanna S, Montassier E, Schmidt B, Patel R, Knights D, Pardi DS,
Kashyap PC. 2016. Gut microbiome predictors of treatment response and
recurrence in primary \emph{Clostridium difficile} infection. Alimentary
Pharmacology \& Therapeutics 44:715--727.

\leavevmode\hypertarget{ref-Pakpour2017}{}%
11. Pakpour S, Bhanvadia A, Zhu R, Amarnani A, Gibbons SM, Gurry T, Alm
EJ, Martello LA. 2017. Identifying predictive features of
\emph{Clostridium difficile} infection recurrence before, during, and
after primary antibiotic treatment. Microbiome 5.

\leavevmode\hypertarget{ref-Lee2020}{}%
12. Lee AA, Rao K, Limsrivilai J, Gillilland M, Malamet B, Briggs E,
Young VB, Higgins PDR. 2020. Temporal gut microbial changes predict
recurrent \emph{Clostridioides difficile} infection in patients with and
without ulcerative colitis. Inflammatory Bowel Diseases
\url{https://doi.org/10.1093/ibd/izz335}.

\leavevmode\hypertarget{ref-Hutton2014}{}%
13. Hutton ML, Mackin KE, Chakravorty A, Lyras D. 2014. Small animal
models for the study of \emph{Clostridium difficile} disease
pathogenesis. FEMS Microbiology Letters 352:140--149.

\leavevmode\hypertarget{ref-Chen2008}{}%
14. Chen X, Katchar K, Goldsmith JD, Nanthakumar N, Cheknis A, Gerding
DN, Kelly CP. 2008. A mouse model of \emph{Clostridium
difficile}-associated disease. Gastroenterology 135:1984--1992.

\leavevmode\hypertarget{ref-Best2012}{}%
15. Best EL, Freeman J, Wilcox MH. 2012. Models for the study of
\emph{Clostridium difficile} infection. Gut Microbes 3:145--167.

\leavevmode\hypertarget{ref-Baxter2014}{}%
16. Baxter NT, Wan JJ, Schubert AM, Jenior ML, Myers P, Schloss PD.
2014. Intra- and interindividual variations mask interspecies variation
in the microbiota of sympatric peromyscus populations. Applied and
Environmental Microbiology 81:396--404.

\leavevmode\hypertarget{ref-Nagpal2018}{}%
17. Nagpal R, Wang S, Woods LCS, Seshie O, Chung ST, Shively CA,
Register TC, Craft S, McClain DA, Yadav H. 2018. Comparative microbiome
signatures and short-chain fatty acids in mouse, rat, non-human primate,
and human feces. Frontiers in Microbiology 9.

\leavevmode\hypertarget{ref-Reeves2011}{}%
18. Reeves AE, Theriot CM, Bergin IL, Huffnagle GB, Schloss PD, Young
VB. 2011. The interplay between microbiome dynamics and pathogen
dynamics in a murine model of \emph{Clostridium difficile} infection
2:145--158.

\leavevmode\hypertarget{ref-Schubert2015}{}%
19. Schubert AM, Sinani H, Schloss PD. 2015. Antibiotic-induced
alterations of the murine gut microbiota and subsequent effects on
colonization resistance against \emph{Clostridium difficile}. mBio 6.

\leavevmode\hypertarget{ref-Jenior2017}{}%
20. Jenior ML, Leslie JL, Young VB, Schloss PD. 2017. \emph{Clostridium
difficile} colonizes alternative nutrient niches during infection across
distinct murine gut microbiomes. mSystems 2.

\leavevmode\hypertarget{ref-Jenior2018}{}%
21. Jenior ML, Leslie JL, Young VB, Schloss PD. 2018. \emph{Clostridium
difficile} alters the structure and metabolism of distinct cecal
microbiomes during initial infection to promote sustained colonization.
mSphere 3.

\leavevmode\hypertarget{ref-Velazquez2019}{}%
22. Velazquez EM, Nguyen H, Heasley KT, Saechao CH, Gil LM, Rogers AWL,
Miller BM, Rolston MR, Lopez CA, Litvak Y, Liou MJ, Faber F, Bronner DN,
Tiffany CR, Byndloss MX, Byndloss AJ, Bäumler AJ. 2019. Endogenous
Enterobacteriaceae underlie variation in susceptibility to
\emph{Salmonella} infection. Nature Microbiology 4:1057--1064.

\leavevmode\hypertarget{ref-Osbelt2020}{}%
23. Osbelt L, Thiemann S, Smit N, Lesker TR, Schröter M, Gálvez EJC,
Schmidt-Hohagen K, Pils MC, Mühlen S, Dersch P, Hiller K, Schlüter D,
Neumann-Schaal M, Strowig T. 2020. Variations in microbiota composition
of laboratory mice influence \emph{Citrobacter rodentium} infection via
variable short-chain fatty acid production. PLOS Pathogens 16:e1008448.

\leavevmode\hypertarget{ref-Stough2016}{}%
24. Stough JMA, Dearth SP, Denny JE, LeCleir GR, Schmidt NW, Campagna
SR, Wilhelm SW. 2016. Functional characteristics of the gut microbiome
in C57BL/6 mice differentially susceptible to \emph{Plasmodium yoelii}.
Frontiers in Microbiology 7.

\leavevmode\hypertarget{ref-Alegre2019}{}%
25. Alegre M-L. 2019. Mouse microbiomes: Overlooked culprits of
experimental variability. Genome Biology 20.

\leavevmode\hypertarget{ref-EtienneMesmin2017}{}%
26. Etienne-Mesmin L, Chassaing B, Adekunle O, Mattei LM, Bushman FD,
Gewirtz AT. 2017. Toxin-positive \emph{Clostridium difficile} latently
infect mouse colonies and protect against highly pathogenic \emph{C.
Difficile}. Gut 67:860--871.

\leavevmode\hypertarget{ref-Lai2020}{}%
27. Lai NY, Musser MA, Pinho-Ribeiro FA, Baral P, Jacobson A, Ma P,
Potts DE, Chen Z, Paik D, Soualhi S, Yan Y, Misra A, Goldstein K,
Lagomarsino VN, Nordstrom A, Sivanathan KN, Wallrapp A, Kuchroo VK,
Nowarski R, Starnbach MN, Shi H, Surana NK, An D, Wu C, Huh JR, Rao M,
Chiu IM. 2020. Gut-innervating nociceptor neurons regulate peyer's patch
microfold cells and SFB levels to mediate \emph{Salmonella} host
defense. Cell 180:33--49.e22.

\leavevmode\hypertarget{ref-Thiemann2017}{}%
28. Thiemann S, Smit N, Roy U, Lesker TR, Gálvez EJC, Helmecke J, Basic
M, Bleich A, Goodman AL, Kalinke U, Flavell RA, Erhardt M, Strowig T.
2017. Enhancement of IFNgamma production by distinct commensals
ameliorates \emph{Salmonella}-induced disease. Cell Host \& Microbe
21:682--694.e5.

\leavevmode\hypertarget{ref-Rolig2013}{}%
29. Rolig AS, Cech C, Ahler E, Carter JE, Ottemann KM. 2013. The degree
of \emph{Helicobacter pylori}-triggered inflammation is manipulated by
preinfection host microbiota. Infection and Immunity 81:1382--1389.

\leavevmode\hypertarget{ref-Ge2018}{}%
30. Ge Z, Sheh A, Feng Y, Muthupalani S, Ge L, Wang C, Kurnick S,
Mannion A, Whary MT, Fox JG. 2018. \emph{Helicobacter pylori}-infected
C57BL/6 mice with different gastrointestinal microbiota have contrasting
gastric pathology, microbial and host immune responses. Scientific
Reports 8.

\leavevmode\hypertarget{ref-Lawley2013}{}%
31. Lawley TD, Young VB. 2013. Murine models to study \emph{Clostridium
difficile} infection and transmission. Anaerobe 24:94--97.

\leavevmode\hypertarget{ref-Buffie2011}{}%
32. Buffie CG, Jarchum I, Equinda M, Lipuma L, Gobourne A, Viale A,
Ubeda C, Xavier J, Pamer EG. 2011. Profound alterations of intestinal
microbiota following a single dose of clindamycin results in sustained
susceptibility to \emph{Clostridium difficile}-induced colitis.
Infection and Immunity 80:62--73.

\leavevmode\hypertarget{ref-Buffie2014}{}%
33. Buffie CG, Bucci V, Stein RR, McKenney PT, Ling L, Gobourne A, No D,
Liu H, Kinnebrew M, Viale A, Littmann E, Brink MRM van den, Jenq RR,
Taur Y, Sander C, Cross JR, Toussaint NC, Xavier JB, Pamer EG. 2014.
Precision microbiome reconstitution restores bile acid mediated
resistance to \emph{Clostridium difficile}. Nature 517:205--208.

\leavevmode\hypertarget{ref-Spinler2016}{}%
34. Spinler JK, Brown A, Ross CL, Boonma P, Conner ME, Savidge TC. 2016.
Administration of probiotic kefir to mice with \emph{Clostridium
difficile} infection exacerbates disease. Anaerobe 40:54--57.

\leavevmode\hypertarget{ref-Markey2018}{}%
35. Markey L, Shaban L, Green ER, Lemon KP, Mecsas J, Kumamoto CA. 2018.
Pre-colonization with the commensal fungus candida albicans reduces
murine susceptibility to \emph{Clostridium difficile} infection. Gut
Microbes 1--13.

\leavevmode\hypertarget{ref-McKee2018}{}%
36. McKee RW, Aleksanyan N, Garrett EM, Tamayo R. 2018. Type IV pili
promote \emph{Clostridium difficile} adherence and persistence in a
mouse model of infection. Infection and Immunity 86.

\leavevmode\hypertarget{ref-Yamaguchi2020}{}%
37. Yamaguchi T, Konishi H, Aoki K, Ishii Y, Chono K, Tateda K. 2020.
The gut microbiome diversity of \emph{Clostridioides
difficile}-inoculated mice treated with vancomycin and fidaxomicin.
Journal of Infection and Chemotherapy 26:483--491.

\leavevmode\hypertarget{ref-Stroke2018}{}%
38. Stroke IL, Letourneau JJ, Miller TE, Xu Y, Pechik I, Savoly DR, Ma
L, Sturzenbecker LJ, Sabalski J, Stein PD, Webb ML, Hilbert DW. 2018.
Treatment of \emph{Clostridium difficile} infection with a
small-molecule inhibitor of toxin UDP-glucose hydrolysis activity.
Antimicrobial Agents and Chemotherapy 62.

\leavevmode\hypertarget{ref-Quigley2019}{}%
39. Quigley L, Coakley M, Alemayehu D, Rea MC, Casey PG, O'Sullivan,
Murphy E, Kiely B, Cotter PD, Hill C, Ross RP. 2019. \emph{Lactobacillus
gasseri} APC 678 reduces shedding of the pathogen \emph{Clostridium
difficile} in a murine model. Frontiers in Microbiology 10.

\leavevmode\hypertarget{ref-Mullish2019}{}%
40. Mullish BH, McDonald JAK, Pechlivanis A, Allegretti JR, Kao D,
Barker GF, Kapila D, Petrof EO, Joyce SA, Gahan CGM, Glegola-Madejska I,
Williams HRT, Holmes E, Clarke TB, Thursz MR, Marchesi JR. 2019.
Microbial bile salt hydrolases mediate the efficacy of faecal microbiota
transplant in the treatment of recurrent \emph{Clostridioides difficile}
infection. Gut 68:1791--1800.

\leavevmode\hypertarget{ref-Nguyen2015}{}%
41. Nguyen TLA, Vieira-Silva S, Liston A, Raes J. 2015. How informative
is the mouse for human gut microbiota research? Disease Models \&
Mechanisms 8:1--16.

\leavevmode\hypertarget{ref-Guh2018}{}%
42. Guh AY, Kutty PK. 2018. \emph{Clostridioides difficile} infection
169:ITC49.

\leavevmode\hypertarget{ref-Tomkovich2019}{}%
43. Tomkovich S, Lesniak NA, Li Y, Bishop L, Fitzgerald MJ, Schloss PD.
2019. The proton pump inhibitor omeprazole does not promote
\emph{Clostridioides difficile} colonization in a murine model. mSphere
4.

\leavevmode\hypertarget{ref-Lawley2012}{}%
44. Lawley TD, Clare S, Walker AW, Stares MD, Connor TR, Raisen C,
Goulding D, Rad R, Schreiber F, Brandt C, Deakin LJ, Pickard DJ, Duncan
SH, Flint HJ, Clark TG, Parkhill J, Dougan G. 2012. Targeted restoration
of the intestinal microbiota with a simple, defined bacteriotherapy
resolves relapsing \emph{Clostridium difficile} disease in mice. PLoS
Pathogens 8:e1002995.

\leavevmode\hypertarget{ref-Wang2019}{}%
45. Wang J, Lang T, Shen J, Dai J, Tian L, Wang X. 2019. Core gut
bacteria analysis of healthy mice. Frontiers in Microbiology 10.

\leavevmode\hypertarget{ref-Lawley2009}{}%
46. Lawley TD, Clare S, Walker AW, Goulding D, Stabler RA, Croucher N,
Mastroeni P, Scott P, Raisen C, Mottram L, Fairweather NF, Wren BW,
Parkhill J, Dougan G. 2009. Antibiotic treatment of \emph{Clostridium
difficile} carrier mice triggers a supershedder state, spore-mediated
transmission, and severe disease in immunocompromised hosts. Infection
and Immunity 77:3661--3669.

\leavevmode\hypertarget{ref-Jump2014}{}%
47. Jump RLP, Polinkovsky A, Hurless K, Sitzlar B, Eckart K, Tomas M,
Deshpande A, Nerandzic MM, Donskey CJ. 2014. Metabolomics analysis
identifies intestinal microbiota-derived biomarkers of colonization
resistance in clindamycin-treated mice. PLoS ONE 9:e101267.

\leavevmode\hypertarget{ref-NagaoKitamoto2020}{}%
48. Nagao-Kitamoto H, Leslie JL, Kitamoto S, Jin C, Thomsson KA,
Gillilland MG, Kuffa P, Goto Y, Jenq RR, Ishii C, Hirayama A, Seekatz
AM, Martens EC, Eaton KA, Kao JY, Fukuda S, Higgins PDR, Karlsson NG,
Young VB, Kamada N. 2020. Interleukin-22-mediated host glycosylation
prevents \emph{Clostridioides difficile} infection by modulating the
metabolic activity of the gut microbiota. Nature Medicine 26:608--617.

\leavevmode\hypertarget{ref-Battaglioli2018}{}%
49. Battaglioli EJ, Hale VL, Chen J, Jeraldo P, Ruiz-Mojica C, Schmidt
BA, Rekdal VM, Till LM, Huq L, Smits SA, Moor WJ, Jones-Hall Y, Smyrk T,
Khanna S, Pardi DS, Grover M, Patel R, Chia N, Nelson H, Sonnenburg JL,
Farrugia G, Kashyap PC. 2018. \emph{Clostridioides difficile} uses amino
acids associated with gut microbial dysbiosis in a subset of patients
with diarrhea. Science Translational Medicine 10:eaam7019.

\leavevmode\hypertarget{ref-Robinson2014}{}%
50. Robinson CD, Auchtung JM, Collins J, Britton RA. 2014. Epidemic
\emph{Clostridium difficile} strains demonstrate increased competitive
fitness compared to nonepidemic isolates. Infection and Immunity
82:2815--2825.

\leavevmode\hypertarget{ref-Collins2015}{}%
51. Collins J, Auchtung JM, Schaefer L, Eaton KA, Britton RA. 2015.
Humanized microbiota mice as a model of recurrent \emph{Clostridium
difficile} disease. Microbiome 3.

\leavevmode\hypertarget{ref-Collins2018}{}%
52. Collins J, Robinson C, Danhof H, Knetsch CW, Leeuwen HC van, Lawley
TD, Auchtung JM, Britton RA. 2018. Dietary trehalose enhances virulence
of epidemic \emph{Clostridium difficile}. Nature 553:291--294.

\leavevmode\hypertarget{ref-Hryckowian2018}{}%
53. Hryckowian AJ, Treuren WV, Smits SA, Davis NM, Gardner JO, Bouley
DM, Sonnenburg JL. 2018. Microbiota-accessible carbohydrates suppress
\emph{Clostridium difficile} infection in a murine model. Nature
Microbiology 3:662--669.

\leavevmode\hypertarget{ref-Fouladi2020}{}%
54. Fouladi F, Glenny EM, Bulik-Sullivan EC, Tsilimigras MCB, Sioda M,
Thomas SA, Wang Y, Djukic Z, Tang Q, Tarantino LM, Bulik CM, Fodor AA,
Carroll IM. 2020. Sequence variant analysis reveals poor correlations in
microbial taxonomic abundance between humans and mice after gnotobiotic
transfer. The ISME Journal
\url{https://doi.org/10.1038/s41396-020-0645-z}.

\leavevmode\hypertarget{ref-Walter2020}{}%
55. Walter J, Armet AM, Finlay BB, Shanahan F. 2020. Establishing or
exaggerating causality for the gut microbiome: Lessons from human
microbiota-associated rodents. Cell 180:221--232.

\leavevmode\hypertarget{ref-Rasmussen2019}{}%
56. Rasmussen TS, Vries L de, Kot W, Hansen LH, Castro-Mejía JL,
Vogensen FK, Hansen AK, Nielsen DS. 2019. Mouse vendor influence on the
bacterial and viral gut composition exceeds the effect of diet. Viruses
11:435.

\leavevmode\hypertarget{ref-Mims2020}{}%
57. Mims TS, Abdallah QA, Watts S, White C, Han J, Willis KA, Pierre JF.
2020. Variability in interkingdom gut microbiomes between different
commercial vendors shapes fat gain in response to diet. The FASEB
Journal 34:1--1.

\leavevmode\hypertarget{ref-Stewart2019}{}%
58. Stewart DB, Wright JR, Fowler M, McLimans CJ, Tokarev V, Amaniera I,
Baker O, Wong H-T, Brabec J, Drucker R, Lamendella R. 2019. Integrated
meta-omics reveals a fungus-associated bacteriome and distinct
functional pathways in \emph{Clostridioides difficile} infection.
mSphere 4.

\leavevmode\hypertarget{ref-Ott2017}{}%
59. Ott SJ, Waetzig GH, Rehman A, Moltzau-Anderson J, Bharti R, Grasis
JA, Cassidy L, Tholey A, Fickenscher H, Seegert D, Rosenstiel P,
Schreiber S. 2017. Efficacy of sterile fecal filtrate transfer for
treating patients with \emph{Clostridium difficile} infection.
Gastroenterology 152:799--811.e7.

\leavevmode\hypertarget{ref-Zuo2017}{}%
60. Zuo T, Wong SH, Lam K, Lui R, Cheung K, Tang W, Ching JYL, Chan PKS,
Chan MCW, Wu JCY, Chan FKL, Yu J, Sung JJY, Ng SC. 2017. Bacteriophage
transfer during faecal microbiota transplantation in \emph{Clostridium
difficile} infection is associated with treatment outcome. Gut
gutjnl--2017--313952.

\leavevmode\hypertarget{ref-Zuo2018}{}%
61. Zuo T, Wong SH, Cheung CP, Lam K, Lui R, Cheung K, Zhang F, Tang W,
Ching JYL, Wu JCY, Chan PKS, Sung JJY, Yu J, Chan FKL, Ng SC. 2018. Gut
fungal dysbiosis correlates with reduced efficacy of fecal microbiota
transplantation in \emph{Clostridium difficile} infection. Nature
Communications 9.

\leavevmode\hypertarget{ref-Robinson2019}{}%
62. Robinson JI, Weir WH, Crowley JR, Hink T, Reske KA, Kwon JH, Burnham
C-AD, Dubberke ER, Mucha PJ, Henderson JP. 2019. Metabolomic networks
connect host-microbiome processes to human \emph{Clostridioides
difficile} infections. Journal of Clinical Investigation 129:3792--3806.

\leavevmode\hypertarget{ref-Fletcher2018}{}%
63. Fletcher JR, Erwin S, Lanzas C, Theriot CM. 2018. Shifts in the gut
metabolome and \emph{Clostridium difficile} transcriptome throughout
colonization and infection in a mouse model. mSphere 3.

\leavevmode\hypertarget{ref-Xiao2015}{}%
64. Xiao L, Feng Q, Liang S, Sonne SB, Xia Z, Qiu X, Li X, Long H, Zhang
J, Zhang D, Liu C, Fang Z, Chou J, Glanville J, Hao Q, Kotowska D,
Colding C, Licht TR, Wu D, Yu J, Sung JJY, Liang Q, Li J, Jia H, Lan Z,
Tremaroli V, Dworzynski P, Nielsen HB, Bäckhed F, Doré J, Chatelier EL,
Ehrlich SD, Lin JC, Arumugam M, Wang J, Madsen L, Kristiansen K. 2015. A
catalog of the mouse gut metagenome. Nature Biotechnology 33:1103--1108.

\leavevmode\hypertarget{ref-Fransen2015}{}%
65. Fransen F, Zagato E, Mazzini E, Fosso B, Manzari C, Aidy SE,
Chiavelli A, D'Erchia AM, Sethi MK, Pabst O, Marzano M, Moretti S,
Romani L, Penna G, Pesole G, Rescigno M. 2015. BALB/c and C57BL/6 mice
differ in polyreactive IgA abundance, which impacts the generation of
antigen-specific IgA and microbiota diversity. Immunity 43:527--540.

\leavevmode\hypertarget{ref-Ivanov2009}{}%
66. Ivanov II, Atarashi K, Manel N, Brodie EL, Shima T, Karaoz U, Wei D,
Goldfarb KC, Santee CA, Lynch SV, Tanoue T, Imaoka A, Itoh K, Takeda K,
Umesaki Y, Honda K, Littman DR. 2009. Induction of intestinal th17 cells
by segmented filamentous bacteria. Cell 139:485--498.

\leavevmode\hypertarget{ref-Azrad2018}{}%
67. Azrad M, Hamo Z, Tkhawkho L, Peretz A. 2018. Elevated serum
immunoglobulin a levels in patients with \emph{Clostridium difficile}
infection are associated with mortality. Pathogens and Disease 76.

\leavevmode\hypertarget{ref-Saleh2019}{}%
68. Saleh MM, Frisbee AL, Leslie JL, Buonomo EL, Cowardin CA, Ma JZ,
Simpson ME, Scully KW, Abhyankar MM, Petri WA. 2019. Colitis-induced
th17 cells increase the risk for severe subsequent \emph{Clostridium
difficile} infection. Cell Host \& Microbe 25:756--765.e5.

\leavevmode\hypertarget{ref-Kozich2013}{}%
69. Kozich JJ, Westcott SL, Baxter NT, Highlander SK, Schloss PD. 2013.
Development of a dual-index sequencing strategy and curation pipeline
for analyzing amplicon sequence data on the MiSeq illumina sequencing
platform. Applied and Environmental Microbiology 79:5112--5120.

\leavevmode\hypertarget{ref-Sze2019}{}%
70. Sze MA, Schloss PD. 2019. The impact of DNA polymerase and number of
rounds of amplification in PCR on 16S rRNA gene sequence data. mSphere
4.

\leavevmode\hypertarget{ref-Schloss2009}{}%
71. Schloss PD, Westcott SL, Ryabin T, Hall JR, Hartmann M, Hollister
EB, Lesniewski RA, Oakley BB, Parks DH, Robinson CJ, Sahl JW, Stres B,
Thallinger GG, Horn DJV, Weber CF. 2009. Introducing mothur:
Open-source, platform-independent, community-supported software for
describing and comparing microbial communities. Applied and
Environmental Microbiology 75:7537--7541.

\leavevmode\hypertarget{ref-Quast2012}{}%
72. Quast C, Pruesse E, Yilmaz P, Gerken J, Schweer T, Yarza P, Peplies
J, Glöckner FO. 2012. The SILVA ribosomal RNA gene database project:
Improved data processing and web-based tools. Nucleic Acids Research
41:D590--D596.

\leavevmode\hypertarget{ref-Cole2013}{}%
73. Cole JR, Wang Q, Fish JA, Chai B, McGarrell DM, Sun Y, Brown CT,
Porras-Alfaro A, Kuske CR, Tiedje JM. 2013. Ribosomal database project:
Data and tools for high throughput rRNA analysis. Nucleic Acids Research
42:D633--D642.

\leavevmode\hypertarget{ref-Westcott2017}{}%
74. Westcott SL, Schloss PD. 2017. OptiClust, an improved method for
assigning amplicon-based sequence data to operational taxonomic units.
mSphere 2.

\leavevmode\hypertarget{ref-Yue2005}{}%
75. Yue JC, Clayton MK. 2005. A similarity measure based on species
proportions. Communications in Statistics - Theory and Methods
34:2123--2131.

\leavevmode\hypertarget{ref-Vegan2018}{}%
76. Oksanen J, Blanchet FG, Friendly M, Kindt R, Legendre P, McGlinn D,
Minchin PR, O'Hara RB, Simpson GL, Solymos P, Stevens MHH, Szoecs E,
Wagner H. 2018. Vegan: Community ecology package.

\leavevmode\hypertarget{ref-r_citation_2018}{}%
77. R Core Team. 2018. R: A language and environment for statistical
computing. R Foundation for Statistical Computing, Vienna, Austria.

\leavevmode\hypertarget{ref-Kuhn2008}{}%
78. Kuhn M. 2008. Building predictive models inRUsing thecaretPackage.
Journal of Statistical Software 28.

\leavevmode\hypertarget{ref-Topcuoglu2020}{}%
79. Topçuoğlu BD, Lesniak NA, Ruffin MT, Wiens J, Schloss PD. 2020. A
framework for effective application of machine learning to
microbiome-based classification problems. mBio 11.

\leavevmode\hypertarget{ref-Tidyverse2019}{}%
80. Wickham H, Averick M, Bryan J, Chang W, McGowan LD, François R,
Grolemund G, Hayes A, Henry L, Hester J, Kuhn M, Pedersen TL, Miller E,
Bache SM, Müller K, Ooms J, Robinson D, Seidel DP, Spinu V, Takahashi K,
Vaughan D, Wilke C, Woo K, Yutani H. 2019. Welcome to the tidyverse.
Journal of Open Source Software 4:1686.

\newpage

\hypertarget{figures}{%
\subsection{Figures}\label{figures}}

\textbf{Figure 1. Microbiota variation is high between mice from
different sources.} A-B. Number of observed OTUs (A) and Shannon
diversity index values (B) across sources of mice at baseline (day -1 of
the experiment). Differences between sources were analyzed by
Kruskal-Wallis test with Benjamini-Hochberg correction for testing each
day of the experiment and the adjusted \emph{P} value was \textless{}
0.05 for panel A (Data Set S1, Sheet 1). None of the \emph{P} values
from pairwise Wilcoxon comparisons between sources were significant
after Benjamini-Hochberg correction (Data Set S1, Sheet 2). Gray lines
represent the median values for each source of mice. C. Principal
Coordinates Analysis (PCoA) of \(\theta_{YC}\) distances of baseline
stool samples. Source and the interaction between source and cage
effects explained most of the variation (PERMANOVA combined
R\textsuperscript{2} = 0.90, \emph{P} \textless{} 0.001; Data Set S1,
Sheet 3). For A-C: each symbol represents the value for a stool sample
from an individual mouse, circles represent experiment 1 mice and
triangles represent experiment 2 mice. D. The median (point) and
interquantile range (colored lines) of the relative abundances for the
20 most significant OTUs out of the 268 OTUs that varied across sources
at baseline by Kruskal-Wallis test with Benjamini-Hochberg correction
(Data Set S1, Sheet 5).

\newpage

\textbf{Figure 2. Clindamycin is sufficient to promote \emph{C.
difficile} colonization in all mice, but clearance time varies across
sources.} A. Setup of the experimental timeline. Mice for the
experiments were obtained from 6 different sources: the Schloss (N = 8)
and Young lab (N = 9) colonies at the University of Michigan, the
Jackson Laboratory (N = 8), Charles River Laboratory (N = 8), Taconic
Biosciences (N = 8), and Envigo (N = 8). Mice that were ordered from
commercial vendors acclimated to the University of Michigan mouse
facility for 13 days prior to antibiotic administration. All mice were
administered 10 mg/kg clindamycin intraperitoneally (IP) 1 day before
challenge with \emph{C. difficile} 630 spores on day 0. Mice were
weighed and feces was collected daily through the end of the experiment
(9 days post-infection). Note: 3 mice died during course of experiment.
1 Taconic mouse prior to infection and 1 Jackson and 1 Envigo mouse
between 1- and 3-days post-infection. B. \emph{C. difficile} CFU/gram
stool measured over time (N = 20-49 mice per timepoint) via serial
dilutions. The black line represents the limit of detection for the
first serial dilution. CFU quantification data was not available for
each mouse due to early deaths, stool sampling difficulties, and not
plating all of the serial dilutions. C. Mouse weight change measured in
grams over time (N = 45-49 mice per timepoint), all mice were normalized
to the weight recorded 1 day before infection. For B-C: timepoints where
differences between sources of mice were statistically significant by
Kruskal-Wallis test with Benjamini-Hochberg correction for testing
across multiple days (Data Set S1, Sheets 6-7) are reflected by the
asterisk above each timepoint (*, \emph{P} \textless{} 0.05). Lines
represent the median for each source and circles represent individual
mice from experiment 1 while triangles represent mice from experiment 2.

\newpage

\textbf{Figure 3. Clindamycin treatment alters bacteria in all sources,
but a subset of bacterial differences across sources persists.} A-B.
Number of observed OTUs (A) and Shannon diversity index values (B)
across sources of mice after clindamycin treatment (day 0). Differences
between sources were analyzed by Kruskal-Wallis test with
Benjamini-Hochberg correction for testing each day of the experiment and
the adjusted \emph{P} value was \textless{} 0.05 (Data Set S1, Sheet 1).
Significant \emph{P} values from the pairwise Wilcoxon comparisons
between sources with Benjamini-Hochberg correction are displayed as the
first initial of each group compared to the group that they are listed
above (Data Set S1, Sheet 2). C. PCoA of \(\theta_{YC}\) distances from
stools collected post-clindamycin. Source and the interaction between
source and cage effects explained most of the variation observed
post-clindamycin (PERMANOVA combined R\textsuperscript{2} = 0.99,
\emph{P} \textless{} 0.001; Data Set S1, Sheet 3). For A-C, each symbol
represents a stool sample from an individual mouse, with circles
representing experiment 1 mice and triangles representing experiment 2
mice. D. The median (point) and interquantile range (colored lines) of
the relative abundances for the 18 OTUs (Data Set S1, Sheet 8) that
varied between sources after clindamycin treatment (day 0). E. The
median (point) and interquantile range (colored lines) of the top 10
most significant OTUs out of 153 with relative abundances that changed
because of the clindamycin treatment (adjusted \emph{P} value
\textless{} 0.05). Data were analyzed by paired Wilcoxon signed rank
test of mice that had paired sequence data for baseline (day -1) and
post-clindamycin (day 0) timepoints (N = 31), with Benjamini-Hochberg
correction for testing all identified OTUs (Data Set S1, Sheet 9). The
gray vertical line indicates the limit of detection.

\newpage

\textbf{Figure 4. Microbiota variation across sources is maintained
after \emph{C. difficile} challenge.} A-B. Number of observed OTUs (A)
and Shannon diversity index values (B) across sources of mice 1-day
post-infection. Data were analyzed by Kruskal-Wallis test with
Benjamini-Hochberg correction for testing each day of the experiment and
the adjusted \emph{P} value was \textless{} 0.05 (Data Set S1, Sheet 1).
Significant \emph{P} values from the pairwise Wilcoxon comparisons
between sources with Benjamini-Hochberg correction are displayed as the
first initial of each group compared to the group that they are listed
above (Data Set S1, Sheet 2). PCoA of \(\theta_{YC}\) distances of 1-day
post-infection stool samples. Source and the interaction between source
and cage effects explained most of the variation between fecal
communities (PERMANOVA combined R\textsuperscript{2} = 0.88, \emph{P}
\textless{} 0.001; Data Set S1, Sheet 3). For A-C: each symbol
represents the value for a stool sample from an individual mouse,
circles represent experiment 1 mice and triangles represent experiment 2
mice. D. The median (point) and interquantile range (colored lines) of
the relative abundances for the top 20 most significant OTUs out of the
44 OTUs that varied between sources 1-day post-infection. The gray
vertical line indicates the limit of detection. For each timepoint OTUs
with differential relative abundances across sources of mice were
identified by Kruskal-Wallis test with Benjamini-Hochberg correction for
testing all identified OTUs (Data Set S1, Sheet 10). E. \(\theta_{YC}\)
distances of fecal samples collected 7-days post-infection relative to
the baseline (day -1) sample for each mouse. Each symbol represents an
individual mouse. Gray lines represent the median for each source.

\newpage

\textbf{Figure 5. Bacteria that influenced whether mice cleared \emph{C.
difficile} by day 7.} A. Post-clindamycin (day 0) relative abundance
data for the 10 OTUs with the highest rankings based on feature weights
in the post-clindamycin (day 0) classification model. Red font
represents OTUs that correlated with \emph{C. difficile} colonization
and blue font represents OTUs that correlated with clearance. Symbols
represent the relative abundance data for an individual mouse. Gray bars
indicate the median relative abundances for each source. The gray
horizontal lines indicate the limit of detection. B. Venn diagram that
combines OTUs that were important to the day -1, 0, and 1 classification
models ( Fig. S4, Data Set S1, Sheet 14) and either overlapped with taxa
that varied across sources at the same timepoint, were impacted by
clindamycin treatment, or both. Bold OTUs were important to more than 1
classification model.

\newpage

\textbf{Figure 6: OTUs associated with \emph{C. difficile} colonization
dynamics vary across sources throughout the experiment.} A-D. Relative
abundances of bold OTUs from Fig. 5B that were important in at least two
classification models are shown over time. A. \emph{Bacteroides} (OTU
2), which varied across sources throughout the experiment. B-C.
\emph{Enterobacteriaceae} (B) and \emph{Enterococcus} (C), which
significantly varied across sources and were impacted by clindamycin
treatment. D. \emph{Porphyromonadaceae} (OTU 7), which was significantly
impacted by clindamycin treatment and after examining relative abundance
dynamics over the course of the experiment was found to also
significantly vary between sources of mice on days -1, 5, 6, 7, and 9 of
the experiment. Symbols represent the relative abundance data for an
individual mouse. Colored lines indicate the median relative abundances
for each source. The gray horizontal line represents the limit of
detection. Timepoints where differences between sources of mice were
statistically significant by Kruskal-Wallis test with Benjamini-Hochberg
correction for testing across multiple days (Data Set S1, Sheet 15) are
identified by the asterisk above each timepoint (*, P \textless{} 0.05).

\newpage

\textbf{Figure S1. Bacterial communities vary between experiments for
some sources.} A-F. PCoA of \(\theta_{YC}\) distances for the baseline
fecal bacterial communities within each source of mice. Each symbol
represents a stool sample from an individual mouse with color
corresponding to experiment and shape representing cage mates.
Experiment number and cage effects explained most of the observed
variation for samples from the Schloss (PERMANOVA combined
R\textsuperscript{2} = 0.99; \emph{P} \(\le\) 0.033) and Young (combined
R\textsuperscript{2} = 0.95; \emph{P} \(\le\) 0.03) mice (Data Set S1,
Sheet 4). G-H: Boxplots of the \(\theta_{YC}\) distances of the 6
sources of mice relative to mice within the same source and experiment
(G) or mice within the same source and between experiments (H) at
baseline (day -1). Symbols represent individual mouse samples: circles
for experiment 1 and triangles for experiment 2.

\newpage

\textbf{Figure S2. \emph{C. difficile} CFU variation across sources
varies slightly between the 2 experiments.} A-B. \emph{C. difficile}
CFU/gram of stool quantification over time for experiment 1 (A) and 2
(B). Experiments were conducted approximately 3 months apart. Lines
represent the median CFU for each source, symbols represent individual
mice and the black line represents the limit of detection. C. \emph{C.
difficile} CFU/gram stool 7-days post-infection across sources of mice
with an asterisk for pairwise Wilcoxon comparisons with
Benjamini-Hochberg correction where \emph{P} \textless{} 0.05. D. Mouse
weight change 2-days post-infection across sources of mice, no pairwise
Wilcoxon comparisons were significant after Benjamini-Hochberg
correction. For C-D: circles represent experiment 1 mice, triangles
represent experiment 2 mice and gray lines indicate the median values
for each group. E. Percent of mice that were colonized with \emph{C.
difficile} over the course of the experiment. Each day the percent is
calculated based on the mice where \emph{C. difficile} CFU was
quantified for that particular day. Total N for each day: day 1 (N =
42), day 2 (N = 20), day 3 (N = 39), day 4 (N = 29), day 5 (N = 43), day
6 (N = 34), day 7 (N = 40), day 8 (N = 36), and day 9 (N = 46).

\newpage

\textbf{Figure S3. Bacterial community composition before, after
clindamycin perturbation, and post-infection can predict \emph{C.
difficile} colonization status 7 days post-infection.} A. Bar graph
visualizations of overall 7-days post-infection \emph{C. difficile}
colonization status that were used as classification outcomes to build
L2-regularized logistic regression models. Mice were classified as
colonized or cleared (not detectable at the limit of detection of 100
CFU) based on CFU g/stool data from 7 days post-infection. B. \emph{C.
difficile} CFU status on Day 7 within each mouse source. N = 8-9 mice
per group. C. L2-regularized logistic regression classification model
area under the receiving operator characteristic curve (AUROCs) to
predict \emph{C. difficile} CFU on day 7 post-infection (Fig. 2B, Fig.
S2C) based on the OTU community relative abundances at baseline (day
-1), post-clindamycin (day 0), and 1-day post-infection. All models
performed better than random chance (AUROC = 0.5, all \emph{P}
\textless{} 0.001, Data Set S1, Sheet 12) and the model built with
post-clindamycin bacterial OTU relative abundances had the best
performance ((\emph{P}\textsubscript{FDR} \textless{} 0.001 for all
pairwise comparisons, Data Set S1, Sheet 13). See Data Set S1, Sheet 14
for list of the 20 OTUs that were ranked as most important to each
model.

\newpage

\textbf{Figure S4. OTUs from classification models based on baseline,
post-clindamycin treatment, or 1-day post-infection community data vary
by source, clindamycin treatment, or both.} A-C. Venn diagrams of OTUs
from the top 20 OTUs from the baseline (A), post-clindamycin treatment
(B), and 1-day post-infection (C) classification models (Data Set S1,
Sheet 14) that overlapped with OTUs that varied across sources at the
corresponding timepoint (Data Set S1, Sheets 5, 8, and 10), were
impacted by clindamycin treatment (Data Set S1, Sheet 9), or both. Bold
OTUs were important to more than 1 classification model.

\newpage

\hypertarget{supplementary-movie-and-data-set-s1}{%
\subsection{Supplementary Movie and Data Set
S1}\label{supplementary-movie-and-data-set-s1}}

\textbf{Movie S1. Large shifts in bacterial community structures
occurred after clindamycin and \emph{C. difficile} infection.} PCoA of
\(\theta_{YC}\) distances animated from days -1 through 9 of the
experiment. Source was the variable that explained the most observed
variation across fecal communities (PERMANOVA source
R\textsuperscript{2} = 0.35, \emph{P} = 0.0001, Data Set S1, Sheet 11)
followed by interactions between cage effects and day of the experiment.
Transparency of the symbol corresponds to the day of the experiment,
each symbol represents a sample from an individual mouse at a specific
timepoint. Circles represent mice from experiment 1 and triangles
represent mice from expeirment 2.

\textbf{Data Set S1, Sheets 1-15. Excel workbook with 15 sheets.}

\textbf{Data Set S1, Sheet 1. Alpha diversity metrics Kruskal-Wallis
statistical results.}

\textbf{Data Set S1, Sheet 2. Alpha diversity metrics pairwise Wilcoxon
statistical results.}

\textbf{Data Set S1, Sheet 3. PERMANOVA results for mice at baseline
(day -1), post-clindamycin (day 0), and post-infection (day 1).}

\textbf{Data Set S1, Sheet 4. PERMANOVA results for each source of mice
at baseline (day -1).}

\textbf{Data Set S1, Sheet 5. OTUs with relative abumdances that
significantly vary between sources at baseline (day -1).}

\textbf{Data Set S1, Sheet 6. \emph{C. difficile} CFU statistical
results.}

\textbf{Data Set S1, Sheet 7. Mouse weight change statistical results.}

\textbf{Data Set S1, Sheet 8. OTUs with relative abundances that
significantly vary between sources post-clindamycin (day 0).}

\textbf{Data Set S1, Sheet 9. OTUs with relative abundances that
significantly changed after clindamycin treatment.}

\textbf{Data Set S1, Sheet 10. OTUs with relative abundances that
significantly vary between sources 1-day post-infection.}

\textbf{Data Set S1, Sheet 11. PERMANOVA results for mice across all
timepoints.}

\textbf{Data Set S1, Sheet 12. Statistical results of L2-regularized
logistic regression model performances compared to random chance.}

\textbf{Data Set S1, Sheet 13. Pairwise comparisons of L2-regularized
logistic regression model performances.}

\textbf{Data Set S1, Sheet 14. Top 20 most important OTUs for each of
the 3 L2-regularized logistic regression models based on OTU relative
abundance data.}

\textbf{Data Set S1, Sheet 15. OTUs with relative abundances that
significantly varied between sources of mice on at least 1 day of the
experiment by Kruskal-Wallis test.}

\end{document}
